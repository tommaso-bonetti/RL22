\documentclass{article}
\usepackage{float}
\usepackage{graphicx}
\usepackage{bigfoot}

\title{Relazione della Prova Finale di Reti Logiche}
\author{Tommaso Bonetti}
\date{1 settembre 2022}

\graphicspath{{res/}}

\renewcommand*\contentsname{Indice}
\renewcommand*\figurename{Figura}
\renewcommand*\tablename{Tabella}

\begin{document}
	\begin{titlepage}
		\maketitle
		\thispagestyle{empty}
	\end{titlepage}

	\newpage
	\tableofcontents
	\newpage

	\section{Introduzione}
		Il progetto di prova finale consiste nell'implementazione in VHDL di un modulo
		hardware che, ricevuta in ingresso una sequenza $U$ di parole di 8 bit, genera in
		uscita un'altra sequenza $Y$ di parole di 8 bit, generate tramite il codificatore
		convoluzionale rappresentato in figura.

		\begin{figure}[H]
			\centering
			\includegraphics[width=.75\linewidth]{conv_des.png}
			\caption{il codificatore convoluzionale da implementare.}
		\end{figure}

		Si può osservare che il convolutore ha tasso di trasmissione $\frac{1}{2}$, ossia
		ogni bit di input viene codificato con 2 bit di output: ne consegue che il numero	di
		parole in output $Z$ sarà il doppio del numero di parole in input $W$.
		Inoltre, il codificatore non elabora parole intere, bensì un flusso di bit
		serializzati $u_k$, generando un flusso di output $y_k = p_{k,1}\|p_{k,2}$ che
		viene concatenato per ottenere le parole da 8 bit che vengono poi scritte in memoria.

		È importante notare come l'elaborazione in sequenza dei bit che compongono le parole
		dell'input fa sì che l'output della codifica di una singola parola sia, in generale,
		variabile, in quanto dipenderà anche dalle parole che sono state codificate prima di
		essa.

		A partire da queste premesse è quindi possibile rappresentare il convolutore come
		una macchina sequenziale sincrona di Mealy, in cui l'output (2 bit) dipende sia dallo
		stato che dall'input (1 bit). La macchina risultante ha 4 stati e si presenta come
		nella figura sotto.

		\begin{figure}[H]
			\centering
			\includegraphics[width=.85\linewidth]{conv_fsm.png}
			\caption{il convolutore come macchina a stati finiti di Mealy.}
		\end{figure}

		\paragraph{Esempio di funzionamento}
		Sia $U = 00100101$, dove il primo bit codificato sarà il Most Significant Bit (in
		questo caso 0). Allora, a partire dall'istante $t = 0$ e ricordando l'ordine di
		concatenazione $y_k = p_{k,1}\|p_{k,2}$, il flusso $y_k$ sarà quello illustrato in
		tabella.

		\begin{table}[H]
			\centering
			\begin{tabular}{ c|c|c|c|c|c|c|c|c }
				$t$ & 0 & 1 & 2 & 3 & 4 & 5 & 6 & 7 \\
				\hline
				$u_k$ & 0 & 0 & 1 & 0 & 0 & 1 & 0 & 1 \\
				$p_{1,k}$ & 0 & 0 & 1 & 0 & 1 & 1 & 0 & 0 \\
				$p_{2,k}$ & 0 & 0 & 1 & 1 & 1 & 1 & 1 & 0 \\
				$y_k$ & 00 & 00 & 11 & 01 & 11 & 11 & 01 & 00
			\end{tabular}
			\caption{i valori di input e di output per ogni istante $0\le t \le 7$.}
		\end{table}

		\noindent Concatenando i bit di $y_k$ si ottiene dunque la sequenza
		$Y = 00001101\ 11110100$.

	\section{Architettura}
		Il progetto è implementato utilizzando tre componenti: una macchina a stati finiti, il
		codificatore convoluzionale e il datapath.

		\subsection{Macchina a stati finiti}
			Si tratta del componente principale, che si interfaccia con il test bench e la
			memoria gestendo l'elaborazione delle sequenze di parole e occupandosi anche di
			coordinare gli altri due componenti e gestire la serializzazione e deserializzazione
			di input e output. Nel codice VHDL è rappresentato dall'entity
			\verb|project_reti_logiche|.

			\paragraph{Porte}
			Come già detto in precedenza questo componente è l'interfaccia verso il test bench,
			motivo per cui le porte sono quelle definite nelle specifiche. Tra esse si segnalano:

			\begin{itemize}
				\item il clock \verb|i_clk|;
				\item i segnali di start e reset \verb|i_start| e \verb|i_rst|;
				\item i vettori di input e output in memoria, rispettivamente \verb|i_data| di 8
					bit e \verb|o_data| di 16 bit;
				\item i segnali di enable e write enable della memoria \verb|o_en| e \verb|o_we|;
				\item il segnale di fine elaborazione \verb|o_done|.
			\end{itemize}

			\paragraph{Architettura}
			L'architettura si interfaccia con i due component \verb|convolutor| e
			\verb|datapath| e descrive due processi, uno sincronizzato e uno combinatorio.

			Il primo, sensibile ai soli segnali di clock e reset, è preposto ad aggiornare lo
			stato della macchina sui fronti di salita del clock e a deserializzare l'output.
			Si noti che il vettore \verb|o_data| viene scritto due bit alla volta, per cui non
			sarebbe possibile assegnare a esso un valore di default in un processo combinatorio
			senza sovrascrivere i dati già codificati: deserializzare in un processo
			combinatorio causerebbe quindi l'inferenza di latch portando ad effetti
			indesiderati, specialmente in post sintesi. Per scongiurare questo rischio risulta
			quindi necessario effettuare la deserializzazione in un processo sincronizzato.

			Il secondo processo, che viene invocato dopo l'aggiornamento dello stato della
			macchina, consiste invece nel serializzare l'input da passare al convolutore e
			nell'aggiornare lo stato prossimo e i segnali necessari per comunicare con il test
			bench e con gli altri moduli.

			Tra i segnali si evidenziano:

			\begin{itemize}
				\item \verb|curr_state| e \verb|next_state|, che rappresentano lo stato attuale e
					lo stato prossimo della macchina;
				\item \verb|dp_rst| e \verb|conv_rst|, che danno il segnale di reset
					rispettivamente a datapath e convolutore;
				\item \verb|dp_done|, che viene alzato a 1 dal datapath quando l'elaborazione è
					terminata e segnala alla macchina a stati che il segnale \verb|o_done| deve
					essere alzato.
			\end{itemize}

			Il significato degli altri segnali verrà spiegato in dettaglio nelle descrizioni
			degli altri moduli.

			\paragraph{Stati}
			La macchina che implementa il modulo ha 17 stati, da $s0$ a $s16$, così organizzati:

			\begin{itemize}
				\item lo stato $s0$ è lo stato iniziale, a cui la macchina viene riportata a ogni
					reset e su cui rimane finché il segnale \verb|i_start| si alza, indicando
					l'inizio di una nuova sequenza di parole da codificare;
				\item lo stato $s1$ legge il numero di parole nella sequenza $U$ dalla prima cella
					di memoria e porta la macchina nel ciclo di codifica di una singola parola;
				\item gli stati $s2$ e $s3$ leggono la successiva parola da codificare;
				\item gli stati da $s5$ a $s12$ eseguono la serializzazione dell'input e la
					deserializzazione dell'output ricevuto dal convolutore;
				\item gli stati da $s13$ a $s15$ scrivono in memoria le due parole di output
					corrispondenti alla codifica dell'input attuale;
				\item lo stato $s16$ riporta la macchina in $s0$ se la sequenza di parole da
					codificare è terminata, altrimenti la porta in $s2$ e ricomincia il ciclo di
					codifica.
			\end{itemize}

			I segnali aggiornati da ogni stato sono illustrati in figura.

			\begin{figure}[H]
				\centering
				\includegraphics[width=\linewidth]{proj_fsm.png}
				\caption{la macchina a stati finiti e i segnali aggiornati da ogni stato.}
			\end{figure}

		\subsection{Convolutore}
			Questo componente implementa il codificatore convoluzionale, che riceve l'input
			serializzato e, per ogni bit, produce due bit di output. Nel codice VHDL è
			rappresentato dall'entity \verb|conv|.

			\paragraph{Porte}
			Ricordando che il convolutore è una macchina sequenziale sincrona, tra le porte si
			trovano:

			\begin{itemize}
				\item il clock \verb|i_clk|;
				\item il segnale di reset \verb|i_rst|, che riporta il convolutore allo stato
					iniziale;
				\item il segnale di enable \verb|i_en|, che permette al convolutore di cambiare
					stato solo quando è alto;
				\item l'ingresso \verb|i_u| e le uscite \verb|o_p1| e \verb|o_p2|.
			\end{itemize}

			\paragraph{Architettura}
			L'architettura, alla stregua di quella della macchina a stati di
			\verb|project_reti_logiche|, descrive un processo sincronizzato e uno combinatorio.

			Il primo, sensibile ai soli segnali di clock e reset, aggiorna lo stato del
			convolutore sui fronti di salita del clock quando il segnale \verb|i_en|
			è alto e riporta il convolutore nello stato iniziale quando \verb|i_rst| è alto.

			Il secondo processo, invocato dopo l'aggiornamento dello stato del convolutore,
			aggiorna lo stato prossimo e le uscite. Si nota che il codificatore è una macchina a
			stati finiti di Mealy, quindi il valore delle uscite dipende sia dallo stato attuale
			che dall'ingresso.

			La macchina ha quattro stati, da $s0$ a $s3$, e il suo funzionamento è illustrato in
			figura 2 (v. pagina 3).

		\subsection{Datapath}
			Questo componente ha il compito di calcolare i corretti indirizzi di memoria per la
			lettura e scrittura dei dati e di indicare quando tutte le parole della sequenza di
			input $U$ sono state codificate e l'elaborazione è terminata. Nel codice VHDL è
			rappresentato dall'entity \verb|dp|.

			\paragraph{Porte}
			Le porte del modulo comprendono:

			\begin{itemize}
				\item i segnali di clock e reset \verb|i_clk| e \verb|i_rst|;
				\item i vettori della parola letta dalla memoria, \verb|i_data|, e della parola da
					scrivere in memoria, \verb|o_data|, entrambi da 8 bit;
				\item il vettore da 16 bit \verb|data_long|, contenente l'output da scrivere in
					memoria;
				\item il vettore da 16 bit \verb|o_address|, che indica l'indirizzo da cui leggere
					o su cui scrivere in memoria;
				\item il segnale \verb|o_done|, che indica che tutte le parole sono state
					codificate e l'elaborazione è terminata;
				\item i segnali \verb|load_raddr| e \verb|load_waddr|, che quando sono alti
					indicano, rispettivamente, di caricare l'indirizzo di lettura o quello di
					scrittura sul vettore \verb|o_address|;
				\item il vettore \verb|sel_out|, che indica quale delle due parole dell'output
					\verb|data_long| deve essere scritta su \verb|o_data|: quando vale 00 non c'è
					output, quando vale 01 viene scritto il primo byte e quando vale 10 viene
					scritto il secondo;
				\item il vettore \verb|rw|, che indica il tipo di accesso alla memoria: quando
					vale 00 non c'è accesso, quando vale 01 c'è accesso in lettura e quando vale 10
					c'è accesso in scrittura.
			\end{itemize}

			\paragraph{Architettura}
			L'architettura fa uso di una serie di segnali, che rappresentano diversi dati di
			interesse. Alcuni vengono aggiornati con processi sincronizzati sul fronte di salita
			del clock, mentre altri sono costantemente aggiornati; il significato di questi
			ultimi è spiegato in maggior dettaglio di seguito:

			\begin{itemize}
				\item \verb|sum_r|, vettore di 16 bit, indica il successivo indirizzo di memoria
					da cui leggere;
				\item \verb|mux_data|, vettore di 8 bit, contiene la successiva parola da scrivere
					in memoria\footnote{Si ricorda che, per ogni parola in input, l'output prodotto
					dal convolutore ha dimensione 16 bit e deve quindi essere diviso in due parole
					di memoria.};
				\item \verb|mux_waddr|, vettore di 16 bit, indica il successivo indirizzo di
					memoria su cui scrivere\footnote{Per ogni parola codificata è necessario
					scrivere due parole in due indirizzi consecutivi, motivo per cui si utilizza
					questo multiplexer la cui uscita corrisponde a uno tra i segnali \verb|sum_w1| e
					\verb|sum_w2|.};
				\item \verb|sub|, vettore di 16 bit, indica il numero di parole rimaste da
					codificare.
			\end{itemize}

			I segnali che vengono aggiornati con processi sincronizzati sono invece i seguenti:

			\begin{itemize}
				\item \verb|read_addr|, che indica l'indirizzo di memoria da cui leggere;
				\item \verb|write_addr|, che indica l'indirizzo di memoria su cui scrivere;
				\item \verb|num_words|, che indica il numero complessivo di parole da codificare.
			\end{itemize}

			Infine, in base ai valori dei segnali interni e a quelli dei segnali in input
			ricevuti dal modulo \verb|project_reti_logiche|, vengono aggiornati i seguenti
			segnali in output:

			\begin{itemize}
				\item \verb|o_data|, che vale 0 se \verb|sel_out| vale 00 e contiene la parola da
					scrivere in memoria se \verb|sel_out| vale 01 o 10;
				\item \verb|o_address|, che vale 0 se \verb|rw| vale 00, contiene l'indirizzo di
					lettura se \verb|rw| vale 01 e contiene l'indirizzo di scrittura se \verb|rw|
					vale 10;
				\item \verb|o_done|, che vale 1 solo se non ci sono più parole da codificare, 0
					altrimenti.
			\end{itemize}

	\newpage

	\section{Risultati sperimentali}

		\subsection{Sintesi}
			\paragraph{Report utilization}
			Di seguito si riporta la tabella Slice Logic. In particolare è possibile notare come
			tutti i registri utilizzati per sintetizzare il modulo siano flip flop.

			\begin{table}[H]
				\centering
				\begin{tabular}{ l|c|c|c|c|c }
					Site Type & Used & Fixed & Prohibited & Available & Util\% \\
					\hline
					Slice LUTs* & 109 & 0 & 0 & 134600 & 0.08 \\
					\quad LUT as Logic & 109 & 0 & 0 & 134600 & 0.08 \\
					\quad LUT as Memory & 0 & 0 & 0 & 46200 & 0.00 \\
					Slice Registers & 83 & 0 & 0 & 269200 & 0.03 \\
					\quad Register as Flip Flop & 83 & 0 & 0 & 269200 & 0.03 \\
					\quad Register as Latch & 0 & 0 & 0 & 269200 & 0.00 \\
					F7 Muxes & 0 & 0 & 0 & 67300 & 0.00 \\
					F8 Muxes & 0 & 0 & 0 & 33650 & 0.00 \\
				\end{tabular}
				\caption{la tabella Slice Logic.}
			\end{table}

			\paragraph{Report timing}
			Il report sul timing evidenzia uno slack di 94.847 ns rispetto a un clock con
			periodo 100 ns.

		\subsection{Simulazioni}
			\subsubsection{Comportamento standard}
				\paragraph{Test bench default}
					Questo test bench contiene due parole da codificare e ha lo scopo di verificare
					la corretta funzionalità del modulo. Questo significa che il modulo deve
					codificare le parole come previsto dalla definizione del convolutore, scrivendo
					i risultati nelle corrette celle di memoria, e gestire il segnale \verb|o_done|
					come richiesto dalla specifica, alzandolo a 1 a fine elaborazione e abbassandolo
					a 0 quando il segnale \verb|i_start| in ingresso dal test bench viene anch'esso
					abbassato.
				\paragraph{Test bench 1 -- funzionalità standard}
					Questo test bench contiene sei parole da codificare e ha lo scopo di verificare
					la corretta funzionalità del modulo, alla stregua del precedente.
				\paragraph{Test bench 2 -- funzionalità standard}
					Questo test bench contiene tre parole da codificare e ha lo scopo di verificare
					la corretta funzionalità del modulo, alla stregua dei precedenti.

			\subsubsection{Corner case}
				\paragraph{Test bench 3 -- sequenza minima}
					Questo test bench contiene zero parole da codificare: ha lo scopo di verificare
					la capacità del modulo di rispondere correttamente a questo input, terminando
					immediatamente l'elaborazione.
				\paragraph{Test bench 4 -- sequenza massima}
					Questo test bench contiene 255 parole da codificare: ha lo scopo di verificare
					la capacità del modulo di rispondere correttamente a questo input, codificando
					correttamente ogni parola della sequenza.
				\paragraph{Test bench 5 -- doppia elaborazione}
					Questo test bench contiene tre parole da codificare per due volte consecutive:
					ha lo scopo di verificare la capacità del modulo di elaborare più sequenze di
					parole senza ricevere il segnale di reset tra esse.
				\paragraph{Test bench 6 -- elaborazione multipla}
					Questo test bench contiene tre sequenze di cinque parole l'una da codificare
					consecutivamente: ha lo scopo di verificare la capacità del modulo di elaborare
					più sequenze di parole senza ricevere il segnale di reset tra esse.
				\paragraph{Test bench 7 -- elaborazione multipla}
					Questo test bench contiene quattro sequenze di tre parole l'una da codificare
					consecutivamente: ha lo scopo di verificare la capacità del modulo di elaborare
					più sequenze di parole senza ricevere il segnale di reset tra esse.
				\paragraph{Test bench 8 -- reset asincrono}
					Questo test bench contiene sei parole da codificare e invia un segnale di reset
					asincrono prima dell'inizio dell'elaborazione: ha lo scopo di verificare la
					capacità del modulo di rispondere correttamente a questo stimolo, iniziando e
					completando correttamente l'elaborazione.

	\section{Conclusioni}
		Il lavoro svolto ha permesso di realizzare un modulo hardware descritto in VHDL in
		grado di codificare una sequenza di parole di 8 bit secondo il codice convoluzionale
		definito nelle specifiche.

		Il prodotto finale è un modulo diviso in tre componenti: la macchina sequenziale
		sincrona, che gestisce la serializzazione e deserializzazione di input e output; il
		convolutore, che codifica il flusso serializzato di bit in input; infine, il datapath,
		che gestisce l'accesso alla memoria e monitora lo stato di avanzamento
		dell'elaborazione.

		Nell'implementazione è stata posta particolare attenzione all'inferenza di latch da
		parte del tool di sintesi di Vivado: per evitare questo comportamento la
		deserializzazione dell'output del convolutore viene eseguita in un processo sincrono.

		Il modulo è stato validato con diversi test bench, che mirano a verificarne la
		funzionalità in condizioni di utilizzo standard e a controllare che il suo
		comportamento rimanga coerente con le specifiche anche nei possibili corner case.
		Tutti i test bench svolti sono stati superati con successo, sia nelle simulazioni
		behavioral che nelle simulazioni funzionali in post sintesi.

\end{document}